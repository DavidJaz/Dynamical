\documentclass[11pt, oneside, article]{memoir}

\settrims{0pt}{0pt} % page and stock same size
\settypeblocksize{*}{35.5pc}{*} % {height}{width}{ratio}
\setlrmargins{*}{*}{1} % {spine}{edge}{ratio}
\setulmarginsandblock{1.1in}{1.4in}{*} % height of typeblock computed
\setheadfoot{\onelineskip}{2\onelineskip} % {headheight}{footskip}
\setheaderspaces{*}{1.5\onelineskip}{*} % {headdrop}{headsep}{ratio}
\checkandfixthelayout

\usepackage[centertags,sumlimits,intlimits,namelimits,reqno]{amsmath}
\usepackage{dutchcal}
\usepackage{latexsym}
\usepackage{minted}
\usepackage{amsfonts,amsthm,amssymb}
\usepackage{mathtools}
\usepackage{outline}
\usepackage[cal=euler,scr=rsfso]{mathalfa}
\usepackage[boxslash]{stmaryrd}
\usepackage{newpxtext}\linespread{1.10}
%\usepackage{exscale}
\usepackage{enumitem}
\usepackage[T1]{fontenc}
%\usepackage{breakcites}
\usepackage[colorlinks,linkcolor=darkblue,citecolor=darkblue,urlcolor=darkblue,breaklinks=true]{hyperref}
\usepackage{tikz}
\usepackage[capitalize]{cleveref}
\usepackage[backend=biber, maxbibnames = 10, style = alphabetic]{biblatex}
\DeclareMathAlphabet{\mathpzc}{OT1}{pzc}{m}{it}
\usepackage[color=white]{todonotes}
\usepackage{xstring}
\usepackage{ebproof}
\usepackage[export]{adjustbox} %vertical align includegraphics
\usepackage{scalefnt}
\usepackage{anyfontsize} %arbitrary font size

%% code from mathabx.sty and mathabx.dcl
\DeclareFontFamily{U}{mathx}{\hyphenchar\font45}
\DeclareFontShape{U}{mathx}{m}{n}{
      <5> <6> <7> <8> <9> <10>
      <10.95> <12> <14.4> <17.28> <20.74> <24.88>
      mathx10
      }{}
\DeclareSymbolFont{mathx}{U}{mathx}{m}{n}
\DeclareFontSubstitution{U}{mathx}{m}{n}
\DeclareMathAccent{\widecheck}{0}{mathx}{"71}

% cleveref %
  \newcommand{\creflastconjunction}{, and\nobreakspace} % serial comma
  

% tikz %
  \usetikzlibrary{ 
  	cd,
  	math,
  	decorations.markings,
		decorations.pathreplacing,
  	positioning,
	 	circuits.logic.US,
 		arrows.meta,
  	shapes,
		shadows,
		shadings,
  	calc,
  	fit,
  	quotes,
  	intersections,
  }

% biblatex %
  \addbibresource{Library20180621.bib} 

% enumitem %
  \setlist{itemsep=-1pt}
	\setlist[description]{leftmargin=0em, itemindent=2em}
	
%-------------------------------------------------------------------------
\theoremstyle{plain}
\newtheorem{theorem}{Theorem}[chapter] %change [] to chapter if we want to change global numbering
\newtheorem{proposition}[theorem]{Proposition}
\newtheorem{corollary}[theorem]{Corollary}
\newtheorem{lemma}[theorem]{Lemma}
\newtheorem{conjecture}[theorem]{Conjecture}

\theoremstyle{definition}
\newtheorem{definition}[theorem]{Definition}
\newtheorem{construction}[theorem]{Construction}
\newtheorem{notation}[theorem]{Notation}
\newtheorem{axiom}{Axiom}
\newtheorem*{axiom*}{Axiom}

\theoremstyle{remark}
\newtheorem{example}[theorem]{Example}
\newtheorem{remark}[theorem]{Remark}
\newtheorem{warning}[theorem]{Warning}

\crefalias{chapter}{section}




%------------------Begin author macros-----------------------

\newcommand{\Set}[1]{\mathrm{#1}}%a named set
\newcommand{\ord}[1]{\underline{#1}}%a natural number, considered as a finite set
\newcommand{\const}[1]{\mathtt{#1}}%a constant, named element of a set, sort of thing
\newcommand{\cat}[1]{\mathcal{#1}}%a generic category
\newcommand{\ccat}[1]{\mathcal{#1}}%a generic bicategory
\newcommand{\Cat}[1]{{\mathsf{#1}}}%a named category
\newcommand{\CCat}[1]{\mathbb{\StrLeft{#1}{1}}\Cat{\StrGobbleLeft{#1}{1}}}%a named bicategory; does not seem to work in section headers...
\newcommand{\funn}[1]{\mathrm{#1}}%a function
\newcommand{\funr}[1]{{#1}}%a generic functor
\newcommand{\Funr}[1]{\mathsf{#1}}%a named functor
\newcommand{\ffunr}[1]{\mathbf{#1}}%a generic 2-functor

\DeclareMathOperator{\ob}{\Set{Ob}}
\DeclareMathOperator{\dom}{dom}
\DeclareMathOperator{\cod}{cod}
\DeclareMathOperator{\Hom}{Hom}
\DeclareMathOperator{\coker}{coker}
\DeclareMathOperator*{\colim}{colim}
\DeclareMathOperator{\im}{im}
\DeclareMathOperator{\inc}{inc}
\DeclarePairedDelimiter{\pair}{\langle}{\rangle}
\DeclarePairedDelimiter{\copair}{[}{]}
\DeclarePairedDelimiter{\classify}{\ulcorner}{\urcorner}
\DeclarePairedDelimiter{\abs}{\lvert}{\rvert}
\DeclarePairedDelimiter{\church}{\llbracket}{\rrbracket}


  \definecolor{darkblue}{rgb}{0,0,0.7} 
  \newcommand{\define}[1]{\emph{#1}}
  \newcommand{\mc}{\mathcal}
  \newcommand{\ot}{\otimes}
  \newcommand{\hooklongrightarrow}{\lhook\joinrel\longrightarrow}
  \DeclareMathOperator\corel{{Corel}}


\newcommand{\rr}{{\mathbb{R}}}
\newcommand{\oo}{{\mathbb{O}}}
\newcommand{\rel}[1][\smset]{\Cat{Rel}_{#1}}
\newcommand{\rrel}[1]{\CCat{Rel}_{#1}}
\newcommand{\prt}{\Funr{Prt}}

\newcommand{\hyp}{\Cat{outer}}
\newcommand{\hcca}[1][-]{A_{#1}}
\newcommand{\cahc}[1][-]{\cat{H}_{#1}}
\newcommand{\alg}{\text{--}\Cat{Alg}}
\newcommand{\aalg}[1][]{\text{--}\CCat{Alg}}
\newcommand{\rgcat}{\Cat{RgCat}}
\newcommand{\rgcalc}{\Cat{RgCalc}}
\newcommand{\rrgcat}[1][]{\CCat{RgCat}}
\newcommand{\regbicat}{\Cat{RgBicat}}

\newcommand{\syn}{\ffunr{syn}}
\newcommand{\prd}{\ffunr{prd}}

\newcommand{\longeq}{=\joinrel=}
\newcommand{\lsh}[2][T]{{#2}_!}%the first argument is unused for now
\newcommand{\ust}[2][T]{{#2}^*}%the first argument is unused for now
\newcommand{\lst}[2][T]{{#2}_*}%the first argument is unused for now

\newcommand{\tn}[1]{\textnormal{#1}}

\newcommand{\id}{\funn{id}}
\newcommand{\finset}{\Cat{FinSet}}
\newcommand{\smset}{\Cat{Set}}
\newcommand{\cospan}{\Cat{Cospan}}
\newcommand{\ccospan}[1][]{\CCat{Cospan}_{#1}}
\newcommand{\cocospan}[1][]{\ccospan[#1]\oop}
\newcommand{\Fam}[1]{\Cat{Fam}_{#1}} 
%\newcommand{\arity}[1][\Lambda]{\Cat{Ar}_{#1}}
%\newcommand{\sspan}[1][\Lambda]{\CCat{Span}\arity[#1]}
\newcommand{\poset}{\Cat{Poset}}
\newcommand{\pposet}{\CCat{Poset}}
\newcommand{\slat}{\Cat{SupLat}}
\newcommand{\ladj}{\Cat{LAdj}}% the category of left adjoints
\newcommand{\radj}{\Cat{RAdj}}% the category of left adjoints
\newcommand{\sslat}{\CCat{SupLat}}
\newcommand{\jslat}{\Cat{JLat}}
\newcommand{\jjslat}{\CCat{JLat}}
\newcommand{\smcat}{\Cat{Cat}}
\newcommand{\ssmcat}{\CCat{Cat}}
\newcommand{\rela}[1]{\Set{Rel}_{#1}}
\newcommand{\func}[1]{\cat{R}_{#1}}
\newcommand{\bij}[1]{\Set{Bij}_{#1}}
\newcommand{\core}{\Cat{Core}}
\newcommand{\grpd}{\Cat{Grpd}}
\newcommand{\disc}{\Cat{Disc}}
\newcommand{\hhyp}{\CCat{outer}}
\newcommand{\of}{\const{OF}}
\newcommand{\io}{\const{io}}
\newcommand{\ff}{\const{ff}}
\newcommand{\thy}{\Funr{Th}}
\newcommand{\str}{\const{str}}
\newcommand{\Th}{\Funr{Th}}
\newcommand{\singleton}{\{1\}}
\newcommand{\true}{\const{true}}
\newcommand{\false}{\const{false}}

\newcommand{\ol}[1]{\overline{#1}}
\newcommand{\comp}{\mathtt{comp}}
\newcommand{\conj}[1]{\widecheck{#1}}

\newcommand{\runitor}{\mathfrak{r}}
\newcommand{\lunitor}{\mathfrak{l}}

\newcommand{\ffrg}{\CCat{FRg}}
\newcommand{\calc}{P}

\newcommand{\zero}{O}

\newcommand{\out}{\mathrm{out}}
\newcommand{\cocolon}{:\!}
\newcommand{\iso}{\cong}
\newcommand{\too}{\longrightarrow}
\newcommand{\tto}{\rightrightarrows}
\newcommand{\To}[1]{\xrightarrow{#1}}
\newcommand{\Too}[1]{\To{\;\;#1\;\;}}
\newcommand{\from}{\leftarrow}
\newcommand{\From}[1]{\xleftarrow{#1}}
\newcommand{\tofrom}{\leftrightarrows}
\newcommand{\surj}{\twoheadrightarrow}
\newcommand{\inj}{\rightarrowtail}
\newcommand{\frsurj}{\twoheadleftarrow}
\newcommand{\frinj}{\leftarrowtail}
\newcommand{\ul}[1]{\underline{#1}}
\renewcommand{\ss}{\subseteq}
\newcommand{\imp}{\Rightarrow}

\newcommand{\op}{^\mathrm{op}}
\newcommand{\co}{^\mathrm{co}}
\newcommand{\inv}{^{-1}}
\newcommand{\tp}{^\dagger}
\newcommand{\mono}{^\mathrm{mono}}
\newcommand{\slice}[1]{_{/#1}}

\newcommand{\cp}{\mathbin{\fatsemi}}
\newcommand{\tens}{\mathbin{\color{gray}\cdot}}
\newcommand{\unit}{\diamond}

\newcommand{\yon}{y}

\newcommand{\powset}{\cat{P}}
\newcommand{\powfin}{\powset_{f}}
\newcommand{\poly}{\Cat{Poly}}
\newcommand{\lens}{\Cat{Lens}}

\newcommand{\nn}{\mathbb{N}}
\newcommand{\zz}{\mathbb{Z}}

\newcommand{\bundle}[2]{\begin{bsmallmatrix}#1\\#2\end{bsmallmatrix}}

\newcommand{\step}[1]{\goodbreak\bigskip\begin{center}------\quad\textbf{#1}\quad------\end{center}\vspace{-.1in}\nopagebreak}
\newcommand{\commentout}[1]{}

\newcommand{\adj}[5][30pt]{%[size] Cat L, Left, Right, Cat R.
\begin{tikzcd}[ampersand replacement=\&, column sep=#1]
  #2\ar[r, shift left=5pt, "{#3}"]\ar[r, phantom, "\Rightarrow" yshift=-.6pt]\&
  #5\ar[l, shift left=5pt, "{#4}"]
\end{tikzcd}
}

\newcommand{\adjr}[5][30pt]{%[size] Cat R, Right, Left, Cat L.
\begin{tikzcd}[ampersand replacement=\&, column sep=#1]
  #2\ar[r, shift left=5pt, "{#3}"]\ar[r, phantom, "\Leftarrow" yshift=-.6pt]\&
  #5\ar[l, shift left=5pt, "{#4}"]
\end{tikzcd}
}

\newcommand{\pb}[1][very near start]{\ar[dr, phantom, #1, "\lrcorner"]}

\def\sub{\begin{outline}\item}
\def\next{\item}
\def\endsub{\end{outline}}

\newcommand{\houroffset}{5}
\newcommand{\minuteoffset}{5}
\newcounter{currentmin}[section]
\newcounter{currenthour}[section]
\setcounter{currentmin}{\minuteoffset}
\setcounter{currenthour}{\houroffset}
\newcommand{\fakeletter}{{\color{white}s}}
\newcommand{\mins}[1]
	{\hfill #1%
	\ifthenelse{\equal{#1}{1}}
		{ min\fakeletter}%
		{ mins}%
	\addtocounter{currentmin}{#1}%
	\ifthenelse{\value{currentmin}>59}%
		{\addtocounter{currentmin}{-60}\addtocounter{currenthour}{1}}%
		{}%
	}
\newcommand{\timecheck}
	{\fakeletter\hfill \fbox{\emph{time check}: 
	\arabic{currenthour}:%
	\ifthenelse{\value{currentmin}<10}
		{0\arabic{currentmin}}
		{\arabic{currentmin}}
	\\
	}}

\begin{document}

\title{Mode dependent dynamical systems and polynomial functors}

\author{David I. Spivak}

%========= Title =========%
\maketitle

\timecheck
\sub Introduction
	\sub Motivation \mins{4}
		\sub Want single formalism for many things:
			\sub Bonding of atoms
			\next Opening and closing of eyes
			\next Busy and ready in a bureaucracy
			\next Connecting to internet using different cell towers
			\endsub
		\next Motto: wiring diagrams, where systems can choose who they wire to based on their internal states
		\endsub
	\next Relation to David Jaz's talks\mins{3}
		\sub Recall
			\sub Generalized lens setup $A\colon\cat{C}\op\to\smcat$
  		\next Apply a variant of the Grothendieck construction to get $\lens_A$
			\next Objects: $(C,A)$, morphisms $(f\colon C\to C', f^\sharp\colon f^*A'\to A)$. 
			\endsub
		\next Today, $\cat{C}\coloneqq\finset$ or $\cat{C}\coloneqq\smset$ and $A\coloneqq\cat{C}/-$.
		\endsub
	\next Plan\mins{3}
		\sub Talk about the category $\poly$
		\next Give an example of using force to break something
		\endsub
	\endsub
\timecheck
\next The category $\poly$
	\sub Definition of $\poly$ \mins{6}
		\sub Full subcategory of $\finset\to\finset$ spanned by sums of representables.
		\next $\poly_{\smset}$ similar, replacing $\finset$ by $\smset$.
			\sub Objects: $P=\sum_{i:P(1)}\yon^{p_i}$
  		\next Morphisms: $\poly(P,Q)=\prod_{i:P(1)}Q(p_i)$.
			\next Morphisms between monomials: lens maps
			\endsub
		\endsub
	\timecheck
	\next Example: stream producers and transducers \mins{4}
		\sub Stream transducers
			\sub Definition: An \emph{$(A,B)$-stream transducer} consists of
  			\sub A set $S$, ``states''
  			\next Functions $r\colon S\to B$ and $u\colon S\to S$.
  			\next Initialized $s_0:S$
  			\next $s_{n+1}=u(s_n)$, $b_n=r(s_n)$.
  			\endsub
			\next Map of polynomials $S\yon^S\to B\yon^A$.
			\endsub
		\next Special case: stream producer
			\sub A \emph{$B$-stream producer} is a $(1, B)$-stream transducer
  		\next Map of polynomials $S\yon^S\to B\yon$.
			\endsub
		\endsub
	\timecheck
	\next Properties of $\poly$
		\sub Has finite products and coproducts given by $0,+,1,\times$ \mins{2}
		\next Has a string of adjoints\mins{4}
\[
\begin{tikzcd}[column sep=60pt]
  \finset
  	\ar[r, shift left=8pt, "n" description]
		\ar[r, shift left=-24pt, "n\yon"']&
  \poly
  	\ar[l, shift right=24pt, "P(0)"']
  	\ar[l, shift right=-8pt, "P(1)" description]
	\ar[l, phantom, "\scriptstyle\Rightarrow"]
	\ar[l, phantom, shift left=16pt, "\scriptstyle\Leftarrow"]
	\ar[l, phantom, shift right=16pt, "\scriptstyle\Rightarrow"]
\end{tikzcd}
\]
and both functors out of $\finset$ are fully faithful (roundtrips on $\finset$ side are isos).
		\next Composition monoidal product: $P\circ Q$. \mins{1}
		\next Has $(\otimes,[-,-])$ adjunction\mins{4}
			\sub $\otimes$ is given by Day convolution of the Cartesian monoidal structure on $\finset$.
				\sub On representables: $\yon^a\otimes\yon^b=\yon^{ab}$
				\next Distributive
				\next So $(3\yon^4+4\yon^2)\otimes(\yon^3+2)=3\yon^{12}+6+4\yon^6+8.$
				\next From a bundle point of view: multiply bundles.
				\endsub
			\next $[P,Q]\coloneqq\prod_{i:P(1)}Q\circ(p_i\yon)$\mins{4}
				\sub Example: $[\yon^n,\yon]=n\yon$ and $[n\yon,\yon]=\yon^n$.
				\next $\poly(P\otimes A,Q)\cong\poly(P,[A,Q])$.
				\endsub
			\endsub
		\endsub
	\endsub
	\timecheck
\next Example
	\sub Simple wiring diagrams
		\sub Mon
		\endsub
	\endsub

\endsub


\begin{minted}{Idris}
S = Work Int Int | Ready
O = Input | Busy | Output Int

TS : S -> Type
I  : O -> Type

TS _         = S
I Input      = Int Int
I Busy       = ()
I (Output _) = ()

Add : Poly (S, TS) -> (O, I)
Add : (s : S) -> (o : O, I o -> TS s)

Add (Work 0 n)     = (Output n, \ ()  -> Ready)
Add (Work (m+1) n) = (Busy,     \ ()  -> Work m (n+1))
Add Ready          = (Input, \ (m, n) -> Work mn) 
\end{minted}




















%\next The category $\poly$
%	\sub As dependent lenses $\bundle{I}{O}$
%		\sub Morphisms are $(f\colon O\to O',f^\sharp\colon f^*I'\to I)$
%		\endsub
%	\next Another formulation: polynomials functors and natural transformations
%		\sub $\bundle{I}{O}$ corresponds to functor $\smset\to\smset$
%		\next Namely, send $X\in\smset$ to $\poly\left(\bundle{X}{1},\bundle{I}{O}\right)$
%		\endsub
%	\next Example: streams in $a$ with carrier $s$ are maps $sx^s\to ax$.
%	\next The functor $P\mapsto P(1)\colon\poly\to\smset$
%		\sub Left adjoint to constant $s\mapsto s$
%			\sub Constant $\smset\to\poly$ is fully faithful
%			\item Unit: $\eta_P\colon P\to P(1)$.
%			\endsub
%		\item Right adjoint to linear $s\mapsto sx$
%			\sub Linear $\smset\to\poly$ is fully faithful
%			\item Counit: $\epsilon_P\colon P(1)x\to P$.
%			\endsub
%		\item Also preserves $\otimes$.
%		\endsub
%	\next Coproducts and products
%  	\sub $\poly(x^n, -)$ commutes with coproducts and products
%		\next We can use this to get a combinatorial formula for morphisms
%		\begin{align*}
%			\poly(P,Q)&=
%			\poly\left(\sum_{i:P(1)}x^{p_i},\;\sum_{j:Q(1)}x^{q_j}\right)\\&=
%			\prod_{i:P(1)}\poly\left(x^{p_i},\;\sum_{j:Q(1)}x^{q_j}\right)\\&=
%			\prod_{i:P(1)}\sum_{j(i):Q(1)}\poly\left(x^{p_i},\;x^{q_j}\right)\\&=
%			\prod_{i:P(1)}\sum_{j(i):Q(1)}\smset(q_j,p_i)
%		\end{align*}
%		\endsub
%	\endsub
%\item Other cool properties
%	\sub All limits
%	\next Another distributive monoidal structure
%		\sub ``Dirichlet product''
%		\next $x^p\otimes x^q=x^{pq}$, distributing over $+$
%		\next Easy to understand in bundle picture: 
%		\endsub
%	\next Another nonsymmetric monoidal structure
%		\sub Composition of functors, composition of polynomials; unit = $x$
%		\next Note that $P\odot 1=P(1)$.
%		\next $(P_1+P_2)\odot Q = (P_1\odot Q)+(P_2\odot Q)$
%		\next $(P\odot Q)R \to P\odot (QR)$, i.e.\ $P\odot-$ has a strength over $\times$
%		\endsub
%	\next The functor $-\odot1\colon\poly\to\smset$
%		\sub Has fully faithful right adjoint $\smset\to\poly$ sending $s\mapsto sx^0$.
%		\next Has fully faithful left adjoint $\smset\to\poly$ sending $s\mapsto sx^1$.
%		\next Preserves $\otimes$.
%		\endsub
%	\next A fully faithful functor $(s\mapsto x^s)\colon\smset\op\to\poly$
%		\sub Strong in three ways
%			\sub Send $+$ to $\times$,
%			\item send $\times$ to $\otimes$, and
%			\item send $\times$ to $\odot$!
%			\endsub
%		\endsub
%	\endsub
%\next Dynamical systems and wiring diagrams
%	\sub Mode-independent = fixed interface = monomial
%	\endsub
%\endsub

\end{document}